\section{Technical Information}

\subsection{Code and Data Availability}

The code produced in the project's scope is available on GitHub:\\
	\url{https://github.com/Zianor/DLIV-chronic-wound-segmentation}.\\
	Package versions are included to ensure reproducibility.

The used data is also available on GitHub:
\url{https://github.com/subbareddy248/WSNET/} \cite{Oota_2021_WACV, Oota_2023_WACV}. Availability at a later point in time cannot be guaranteed.

\subsection{Libraries}

Several libraries were used in this project. All work is based on the Deep Learning framework TensorFlow with Keras \cite{tensorflow2015-whitepaper, chollet2015keras}. The implementation of the four used network architectures was provided by the Python library \texttt{segmentation\_models} \cite{SegmentationModels}. Image augmentations were performed with \texttt{Albumentations} \cite{albumentations}.

\subsection{Used Hardware}

All computations are performed on a MacBook Air (24\,GB RAM, Apple M2 Chip with an 8-core GPU) or a computer with 16GB RAM and an Nvidia GeForce GTX 1070 Ti as GPU. The package versions for GPU-utilization on MacOS are included in the package versions on GitHub.

\subsection{Prior Experience}

I have a strong programming background, consisting of a B.Sc. in Computer Science and three years of experience in Web development with Python. Besides the Advanced Concepts of Machine Learning course, I have no experience with Deep Learning.

\subsection{Learning Process}

\begin{itemize}
	\item Getting familiar with tensorflow
	\item learning about the state of the art in segmantic segmentation and segmentation of wound images and evaluation methods
	\item more experience in dealing with paper results and how trustworthy they are
	\item first, I planned on spending more time on the results of the chosen paper and experimenting with different augmentations and explainability
	\item after doing research about segmentation models I got sceptical about the general approach and spend a lot of time in researching semantic segmentation and the how and why
	\item 
\end{itemize}
