\section{Technical Information}

\subsection{Prior Experience}

I have a strong programming background, consisting of a B.Sc. in Computer Science and three years of work experience in Web Developement with Python. Beside the content of the course Advanced Concepts of Machine Learning, I have no prior experience with Deep Learning.

\subsection{Code and Data Availability}

The code produced in the scope of the project is available on GitHub: \url{https://github.com/Zianor/DLIV-chronic-wound-segmentation}. Package versions are included to ensure reproducability.

The used data is available on GitHub as well: \url{https://github.com/subbareddy248/WSNET/} \cite{Oota_2021_WACV, Oota_2023_WACV}. Availability on a later point of time cannot be guaranteed.

\subsection{Libraries}

Several libraries were used in this project. The Deep Learning framework all work is based on is Tensorflow with Keras \cite{tensorflow2015-whitepaper, chollet2015keras}. The implementation of the 4 used network architectures was provided by the Python library \texttt{segmentation\_models} \cite{SegmentationModels}. Image augmentations were performed with \texttt{Albumentations} \cite{albumentations}.

\subsection{Learning Process}

\begin{itemize}
	\item Getting familiar with tensorflow
	\item learning about the state of the art in segmantic segmentation and segmentation of wound images and evaluation methods
	\item more experience in dealing with paper results and how trustworthy they are
	\item first, I planned on spending more time on the results of the chosen paper and experimenting with different augmentations and explainability
	\item after doing research about segmentation models I got sceptical about the general approach and spend a lot of time in researching semantic segmentation and the how and why
	\item 
\end{itemize}


\subsection{Used Hardware}

All computations are performed on one of two different machines: a MacBook Air (24\,GB RAM, Apple M2 Chip with an 8-core GPU) or a computer with 16GB and a nvidia GeForce GTX 1070 Ti as GPU. The package versions for GPU-utilization on MacOS are included in the package versions on GitHub.

