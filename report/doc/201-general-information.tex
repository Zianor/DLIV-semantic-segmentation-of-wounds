\section{Technical Information}

\subsection{Code and Data Availability}

The code produced in the project's scope is available on GitHub: \url{https://github.com/Zianor/DLIV-semantic-segmentation-of-wounds}. Package versions are included to ensure reproducibility.

The used data is also available on GitHub:
\url{https://github.com/subbareddy248/WSNET/} \cite{Oota_2021_WACV, Oota_2023_WACV}. Availability at a later point in time cannot be guaranteed.

\subsection{Libraries}

Several libraries were used in this project. All work is based on the Deep Learning framework TensorFlow with Keras \cite{tensorflow2015-whitepaper, chollet2015keras}. The implementation of the four used network architectures was provided by the Python library \texttt{segmentation\_models} \cite{SegmentationModels}. Image augmentations were performed with \texttt{Albumentations} \cite{albumentations}.

\subsection{Used Hardware}

All computations are performed on a MacBook Air (24\,GB RAM, Apple M2 Chip with an 8-core GPU) or a computer with 16GB RAM and an Nvidia GeForce GTX 1070 Ti as GPU. The package versions for GPU-utilization on MacOS are included in the package versions on GitHub.

\subsection{Prior Experience}

I have a strong programming background, consisting of a B.Sc. in Computer Science and three years of experience in Web development with Python. Besides the Advanced Concepts of Machine Learning course, I have no experience with Deep Learning.

\subsection{Learning Process}

During this project I got familiar with TensorFlow and Keras. I learned a lot about the state-of-the-art methods for semantic segmentation and segmentation of wound images in particular and how they can be evaluated. I was not familiar with the local contextualisation of features that is done by using different encoder-decoder architectures and I found it particularly interest combined with the WSNet paper. Originally, I planned on spending more time with explainability of segmentation results but frameworks I found were written mainly for PyTorch. Additionally, the proposed architecture of WSNet sounded better before I knew about the state of the art in segmentation. So I decided to focus on global and local features in semantic segmentation and spend a lot of time researching the background of the model architectures and what their abilities and limitations are.

I was also suprised again how the initial impression can differ from more detailed assessment of methods. The paper was well-written, included many details and generally sounded trustworthy. Only the implementation itself showed that many details differed from the description and that the idea forming the base of their work was not fully-thought through or at least not motivated sufficiently.