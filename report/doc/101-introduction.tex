\section{Introduction}

\subsection{Motivation}

Many people are affected by chronic wounds that need to be monitored to detect infections and ensure proper healing of the wound \cite{DFUC2022}. Other wounds, e.g., surgical wounds need to be monitored as well. Such monitoring might be necessary over long time span \cite{DFUC2022}. Monitoring by visual assesment yields the risk of missing changes and therefore wound segmentation based on images is needed. Without an algorithm to automise this process, a manual segmentation by experts is needed. However, manual segmentation is not only very expensive and time consuming but experts also differ in the provided segmentation.

An automisation is complex due to various reasons. On one side, the characteristics of wounds itself hold challenges: Wounds have a complex structure and contain different types of tissues with different colours and textures \cite{AhmadFauzi2015}. This means, there are not only borders between wound and healthy tissue but also in the wound itself. There exist also many different types of wounds, e.g. diabetic foot ulcers, surgical wounds, decubitus and many more that have different characteristcs. On the other side, the images itself are very heterogenous. lighting conditions, the distance to the camera, the camera angle and event the camera itself change the images in a way that they have an impact on the resulting segmentation. Creating a controlled environment sounds like the desired solution but is not feasible in a clinical setting. Ideally, it should be possible to take pictures with a smartphone without overly complicated instructions for the person taking the picture. A clinical employee without further experience as photographer should be able to take pictures that are then segmented correcty.

\subsection{Research Questions}

A recent publication by \citeauthor{Oota_2023_WACV} claims to have improved the state of the art in the field of Wound Segmentation. Such claims always need to be supported by further research. This project aims to investigate and reimplement the proposed method. Furthermore, the method is set into context with state-of-the-art methods for semantic segmentation in general and for wounds specifically. In a second step, the robustness of the models is assessed and evaluated whether the learned features are generally transferable and robust to transformations on the input.

%\begin{itemize}
%	\item can the results be reproduced?
%	\item what influence does the input image size have? Can we rescale the images and are able to transfer what is learned
%	\item how robust is the model/architectures to transformations/distortions on the input
%	\item XAI
%\end{itemize}