\section{Introduction}

\subsection{Motivation}

Many people are affected by chronic wounds that need to be monitored to detect infections and ensure proper healing of the wound \cite{DFUC2022}. Other wounds, e.g., surgical wounds, also need to be monitored. Such monitoring might be necessary over a long period \cite{DFUC2022}. Monitoring by visual assessment yields the risk of missing changes, so wound segmentation based on images is required. Manual segmentation by experts is needed without an algorithm to automate this process. However, manual segmentation is very expensive and time-consuming, and experts differ in the provided segmentation.

Automatisation is complex due to various reasons. On one side, the characteristics of wounds themselves hold challenges: Wounds have a complex structure and contain different types of tissues with different colours and textures \cite{AhmadFauzi2015}. This means there are borders between the wound and healthy tissue and the wound itself. There are also many different types of wounds, e.g., diabetic foot ulcers, surgical wounds, pressure ulcers, and many others, with different characteristics. On the other hand, the images themselves are very heterogeneous. Lighting conditions, the distance to the camera, the camera angle, and even the camera itself change the images in a way that impacts the resulting segmentation. Creating a controlled environment sounds like the desired solution but is not feasible in a clinical setting. Ideally, it should be possible to take pictures with a smartphone without overly complicated instructions for the person taking the picture. A clinical employee without further experience as a photographer should be able to take images that are then segmented correctly.

\subsection{Research Questions}

A recent publication by \citeauthor{Oota_2023_WACV} claims to have improved state of the art in the field of Wound Segmentation. Such claims always need to be supported by further research. This project aims to investigate and reimplement the proposed method. Furthermore, the method is contextualised with state-of-the-art general semantic and wound segmentation methods. In the next step, the robustness of the models is assessed and evaluated to determine whether the learned features are generally transferable and robust to transformations on the input.

%\begin{itemize}
%	\item can the results be reproduced?
%	\item What influence does the input image size have? Can we rescale the images and transfer what is learned
%	\item how robust is the model/architectures to transformations/distortions on the input
%	\item XAI
%\end{itemize}