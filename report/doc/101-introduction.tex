\section{Introduction}

\subsection{Motivation}

\begin{itemize}
	\item many people affected by chronic wounds that need to be monitored
	\item Why is automatic Wound Segmentation so important? And why is it a complex problem?
	\item Manual segmentation by experts expensive and very time consuming
	\item experts differ in their segmentation
	\item different types of wounds have different characteristics
	\item changing lighting conditions, distance to camera, camera angle, different cameras have impact on result
	\item controlled environment not feasible in clinical setting
	\item ideally, we want to be able to take pictures with a smartphone without overly complicated instructions for the person taking the picture
	\item experience as photographer should not be required, clinical proffessionals should be able to take pictures that are then segmented correctly
\end{itemize}

% What are wounds that need to be segmented? Focus on chronic wounds
\begin{itemize}
	\item one type: diabetic foot ulcers $\rightarrow$ are monitored to ensure healing process is optimal and there is no infection, normally long time span \cite{DFUC2022}
\end{itemize}

% Why is wound segmentation a more complex segmentation problem
\begin{itemize}
	\item wounds have complex structure containing different types of tissue with different colour and texture $\rightarrow$ different regions with borders in between \cite{AhmadFauzi2015}
	\item heterogeneous wound images
\end{itemize}


%% What are wounds that need to be segmented? Focus on chronic wounds
%\begin{itemize}
%	\item one type: diabetic foot ulcers $\rightarrow$ are monitored to ensure healing process is optimal and there is no infection, normally long time span \cite{DFUC2022}
%\end{itemize}
%
%% Why is wound segmentation a more complex segmentation problem
%\begin{itemize}
%	\item wounds have complex structure containing different types of tissue with different colour and texture $\rightarrow$ different regions with borders in between \cite{AhmadFauzi2015}
%	\item heterogeneous wound images
%\end{itemize}


\subsection{Research Questions}

A recent publication by \citeauthor{Oota_2023_WACV} claims to have improved the state of the art in the field of Wound Segmentation. Such claims always need to be supported by further research. This project aims to investigate and reimplement the proposed method. Furthermore, the method is set into context with state-of-the-art methods for semantic segmentation in general and for wounds specifically. 
\begin{itemize}
	\item can the results be reproduced?
	\item what influence does the input image size have? Can we rescale the images and are able to transfer what is learned
	\item how robust is the model/architectures to transformations/distortions on the input
	\item XAI
\end{itemize}